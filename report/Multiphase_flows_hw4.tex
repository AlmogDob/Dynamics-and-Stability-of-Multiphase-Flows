\documentclass[11pt, a4paper]{article}

\usepackage{amsmath, amssymb, titling}
\usepackage[margin=2.5cm]{geometry}
\usepackage[colorlinks=true, linkcolor=black, urlcolor=black, citecolor=black]{hyperref}
\usepackage{graphicx}
\usepackage{caption}
\usepackage{subcaption}
\usepackage{float}
\usepackage{cancel}
\usepackage{fancyhdr, lastpage}
\usepackage{fourier-orns}
\usepackage{xcolor}
\usepackage{matlab-prettifier}

\renewcommand\maketitlehooka{\null\mbox{}\vfill}
\renewcommand\maketitlehookd{\vfill\null}
\renewcommand{\headrule}{
\vspace{-5pt}\hrulefill
\raisebox{-2.1pt}{\quad\leafleft\decoone\leafright\quad}\hrulefill}

\title{Dynamics and Stability of Multiphase Flows \\ HW4}
\author{Almog Dobrescu\\\\ID 214254252}

\pagestyle{fancy}
\cfoot{Page \thepage\ of \pageref{LastPage}}

\begin{document}

\maketitle

\thispagestyle{empty}
\newpage
\setcounter{page}{1}

\tableofcontents
\vfil
\listoffigures
\newpage

\section{Velocity Profile $U_{\left(z\right)}$ as a Funciton of \emph{z}}
Consider a liquid jet ejected from a circular orifice of radius \emph{a} (point A). The jet has flux \emph{Q} of fluid with density $\rho$ that exits with velocity $U_0$. The surface tension between the fluid and the air with pressure $P_0$ is $\sigma$. \\
By assuming laminar, inviscid, incpmressible, axisymmetric and steady flow, we get the Bernoulli equation, between point A and arbitrary point downstream B:
\begin{equation}
    \frac{1}{2}\rho U_0^2+\rho gz+P_A=\frac{1}{2}\rho U_{\left(z\right)}^2+P_B
\end{equation}
The Laplace pressure for a jet:
\begin{equation}
    \begin{array}{c}
        \begin{matrix}
            \displaystyle\Delta P=\sigma\left(\frac{1}{R_1}+\frac{1}{R_2}\right) && \left\{\begin{array}{l}
                R_1=R_{1\left(z\right)}=\text{inner radius (normal to the flow)} \\
                R_2=R_{2\left(z\right)}=\text{outer radius (parallel to the flow)}
            \end{array}\right.
        \end{matrix} \\\\
        \Downarrow \\
        \begin{matrix}
            \displaystyle P_A=P_0+\sigma\left(\frac{1}{a}\right) && \displaystyle P_B=P_0+\sigma\left(\frac{1}{R_1}+\frac{1}{R_2}\right)
        \end{matrix}
    \end{array}
\end{equation}
substituting into the Bernoulli equation:
\begin{equation}
    \frac{1}{2}\rho U_0^2+\rho gz+\cancel{P_0}+\frac{\sigma}{a}=\frac{1}{2}\rho U_{\left(z\right)}^2+\cancel{P_0}+\sigma\left(\frac{1}{R_1}+\frac{1}{R_2}\right)
\end{equation}
Isolating $U_{\left(z\right)}$ we get:
\begin{equation}
    \frac{U_{\left(z\right)}}{U_0}=\sqrt{1+\frac{2gz}{U_0^2}+\frac{2\sigma}{\rho U_0^2}\left(\frac{1}{a}-\frac{1}{R_1}-\frac{1}{R_2}\right)}
\end{equation}
and in dimensionless form:
\begin{equation}
    \frac{U_{\left(z\right)}}{U_0}=\sqrt{1+\underbrace{\frac{2z}{Fr^2a}}_{\substack{\text{acceleration}\\\text{due to gravity}}}+\underbrace{\frac{2}{We}\left(1-\frac{a}{R_1}-\frac{a}{R_2}\right)}_{\substack{\text{deceleration}\\\text{due to surface tension}}}}
\end{equation}
where:
\begin{itemize}
    \item $\displaystyle We=\frac{\rho U_0^2a}{\sigma}$ is the Webber \#
    \item $\displaystyle Fr=\frac{U_0^2}{ga}$ is the Froud \#
\end{itemize}
Let's consider the curvature of the wall of the get parallel to the flow:
\begin{equation}
    \underbrace{\kappa}_{\substack{\text{principal}\\\text{curvature}}}=-\frac{R_{1zz}}{\left(1+R_{1z}^2\right)^{\frac{3}{2}}}\underbrace{=}_{\text{linearization}}-R_{1zz}
\end{equation}
The principal curvature equals one over the principal radius so:
\begin{equation}
    \kappa=\frac{1}{R_2}=-\frac{\partial^2 R_1}{\partial z^2}
\end{equation}
So, the velocity profile $U_{\left(z\right)}$ as a funciton of \emph{z} is therefore:
\begin{equation}
    \colorbox{yellow}{$\displaystyle\frac{U_{\left(z\right)}}{U_0}=\sqrt{1+\frac{2z}{Fr^2a}+\frac{2}{We}\left(1-\frac{a}{R_1}+a\frac{\partial^2 R_1}{\partial z^2}\right)}$}
\end{equation}

\newpage

\section{ODE of Jet Radius $r_{\left(z\right)}$}
\subsection{Formulate the ODE}
From conservation of flux:
\begin{equation}
    \begin{array}{c}
        \displaystyle Q=2\pi\int_0^{R_1}U_{\left(z\right)}rdr=\pi R_1^2U_{\left(z\right)}\underbrace{=}_{z=0}\pi a^2U_0 \\\\
        Q_{\left(z\right)}=\pi U_{\left(z\right)}R_1^2\underbrace{=}_{\substack{\text{conservation}\\\text{of mass}}}\pi a^2U_0 \\\\
        \Downarrow \\
        \displaystyle \frac{R_1}{a}=\sqrt{\frac{U_0}{U_{\left(z\right)}}}
    \end{array}
\end{equation}
After substituting $U_{\left(z\right)}$:
\begin{equation}
    \colorbox{yellow}{$\displaystyle R_1=\frac{a^2U_0}{\displaystyle\sqrt{U_0^2+2gz+\frac{2\sigma}{\rho}\left(\frac{1}{a}-\frac{1}{R_1}+\frac{\partial^2 R_1}{\partial z^2}\right)}}$}
\end{equation}

\subsection{Boundary Conditions}
Assume BC:
\begin{enumerate}
    \item $R_{1\left(z=0\right)}=a$
    \item $\displaystyle\left.\frac{\partial R_1}{\partial z}\right|_{z=0}=0$
\end{enumerate}
The first conditions is given. \\
The second condition is derived from the fact that at the begining, the jet looks like a cylinder.
\newpage

\section{Numerical Solver}
The selected method is the Runge-Kutta-Merson method. \\
Isolating the second derivitive of $R_1$:
\begin{equation}
    \begin{array}{rcl}
        \displaystyle\frac{a^4}{R_1^4}&=&\displaystyle1+\frac{2z}{\text{Fr}^2a}+\frac{2}{\text{We}}\left(1-\frac{a}{R_1}+a\frac{\partial^2 R_1}{\partial z^2}\right) \\\\
        \displaystyle\frac{\text{We}\cdot a^4}{2R_1^4}&=&\displaystyle\frac{\text{We}}{2}+\frac{\text{We}z}{\text{Fr}^2a}+1-\frac{a}{R_1}+a\frac{\partial^2R_1}{\partial z^2} \\\\
        \displaystyle\frac{\partial^2R_1}{\partial z^2}&=&\displaystyle\frac{\text{We}\cdot a^3}{2R_1^4}-\frac{\text{We}}{2a}-\frac{\text{We}\cdot z}{\text{Fr}^2a^2}-\frac{1}{a}+\frac{1}{R_1}
    \end{array}
\end{equation}
% Where:
% \begin{equation}
%     \begin{array}{rcl}
%         \displaystyle\left.\frac{\partial^2R_1}{\partial z^2}\right|_i=\displaystyle f^{''}_i&=&\displaystyle\frac{\delta^2f_i}{h^2} \\\\
%         &=&\displaystyle\frac{\delta\left(f_{i+\frac{1}{2}}-f_{i-\frac{1}{2}}\right)}{h^2} \\\\
%         &=&\displaystyle\frac{\left(f_{i+1}-2f_{i}+f_{i-1}\right)}{h^2}
%     \end{array}
% \end{equation}
% So the ODE becomes:
% \begin{equation}
%     R_{1,i+1}=\left(\displaystyle\frac{\text{We}\cdot a^3}{2R_1^4}-\frac{\text{We}}{2a}-\frac{\text{We}z}{\text{Fr}^2a^2}-\frac{1}{a}+\frac{1}{R_1}\right)h^2+2R_{1,i}-R_{1,i-1}
% \end{equation}
Defining the system of equations:
\begin{equation}
    \begin{matrix}
        x=R_1 &&& y=\displaystyle\frac{\partial R_1}{\partial z}
    \end{matrix}
\end{equation}
\begin{equation}
    \begin{matrix}
        \left\{\begin{array}{rcl}
            \displaystyle\frac{\partial x}{\partial z}&=&g_{\left(z,x,y\right)} \\\\
            \displaystyle\frac{\partial y}{\partial z}&=&f_{\left(z,x,y\right)}
        \end{array}\right. &\text{where}& \begin{array}{rcl}
            g&=&y \\\\
            f&=&\displaystyle\frac{\text{We}\cdot a^3}{2x^4}-\frac{\text{We}}{2a}-\frac{\text{We}\cdot z}{\text{Fr}^2a^2}-\frac{1}{a}+\frac{1}{x}
        \end{array}
    \end{matrix}
\end{equation}
The Runge-Kutta-Merson method is defined as:
\begin{equation}
    \begin{array}{l}
        \left\{\begin{array}{rcl}
            x_{i+1}&=&\displaystyle x_i+\frac{1}{6}\left(m_1+4m_4+m_5\right) \\\\
            y_{i+1}&=&\displaystyle y_i+\frac{1}{6}\left(k_1+4k_4+k_5\right)
        \end{array}\right. \\\\
        \left\{\begin{array}{rcl}
            m_1&=&hg_{\left(z_i, x_i, y_i\right)} \\
            k_1&=&hf_{\left(z_i, x_i, y_i\right)}
        \end{array}\right. \\\\
        \left\{\begin{array}{rcl}
            m_2&=&hg_{\left(z_i+\frac{1}{3}h, x_i+\frac{1}{3}m_1, y_i+\frac{1}{3}k_1\right)} \\
            k_2&=&hf_{\left(z_i+\frac{1}{3}h, x_i+\frac{1}{3}m_1, y_i+\frac{1}{3}k_1\right)}
        \end{array}\right. \\\\
        \left\{\begin{array}{rcl}
            m_3&=&hg_{\left(z_i+\frac{1}{3}h, x_i+\frac{1}{6}\left(m_1+m_2\right), y_i+\frac{1}{6}\left(k_1+k_2\right)\right)} \\
            k_3&=&hf_{\left(z_i+\frac{1}{3}h, x_i+\frac{1}{6}\left(m_1+m_2\right), y_i+\frac{1}{6}\left(k_1+k_2\right)\right)} 
        \end{array}\right. \\\\
        \left\{\begin{array}{rcl}
            m_4&=&hg_{\left(z_i+\frac{1}{2}h, x_i+\frac{1}{8}\left(m_1+3m_3\right), y_i+\frac{1}{8}\left(k_1+3k_3\right)\right)} \\
            k_4&=&hf_{\left(z_i+\frac{1}{2}h, x_i+\frac{1}{8}\left(m_1+3m_3\right), y_i+\frac{1}{8}\left(k_1+3k_3\right)\right)} 
        \end{array}\right. \\\\
        \left\{\begin{array}{rcl}
            m_5&=&hg_{\left(z_i+h, x_i+\frac{1}{2}\left(m_1-3m_3+4m_4\right), y_i+\frac{1}{2}\left(k_1-3k_3+4k_4\right)\right)} \\ 
            k_5&=&hf_{\left(z_i+h, x_i+\frac{1}{2}\left(m_1-3m_3+4m_4\right), y_i+\frac{1}{2}\left(k_1-3k_3+4k_4\right)\right)} 
        \end{array}\right.
    \end{array}
\end{equation}

\newpage

\section{Solution}
\begin{equation}
    \begin{matrix}
        \displaystyle U_0=1\left[\frac{\mathrm{m}}{\mathrm{sec}}\right] & a=0.05\left[\mathrm{m}\right] & \displaystyle\rho=998\left[\frac{\mathrm{kg}}{\mathrm{m}^3}\right] & \displaystyle\sigma=0.0728\left[\frac{\mathrm{N}}{\mathrm{m}}\right] & \displaystyle g=9.81\left[\frac{\mathrm{m}}{\mathrm{sec}^2}\right] \\\\
        & h=10^{-5}\left[m\right] && z=\left[0,h,2h,\cdots,5\right]\left[m\right]
    \end{matrix}
\end{equation}
\begin{figure}[H]
    \centering
    \includegraphics[width=0.9\textwidth]{images/graph1.png}
    \caption{The jet shape}
    \label{fig:The_jet_shape}
\end{figure}
\begin{figure}[H]
    \centering
    \includegraphics[width=0.9\textwidth]{images/graph2.png}
    \caption{$R_1$ as a function of z}
    \label{fig:R_1_of_z}
\end{figure}
\begin{figure}[H]
    \centering
    \begin{subfigure}[c]{0.49\textwidth}
        \centering
        \includegraphics[width=\textwidth]{images/graph2.1.png}
        \caption{$R_1$ as a function of z - zoomed on begining}
        \label{fig:R_1_of_z_zoom_on_begining}
    \end{subfigure}
    \begin{subfigure}[c]{0.49\textwidth}
        \centering
        \includegraphics[width=\textwidth]{images/graph2.2.png}
        \caption{$R_1$ as a function of z - zoomed on end}
        \label{fig:R_1_of_z_zoom_on_end}
    \end{subfigure}
    \caption{$R_1$ as a funciton of z - zoomed}
        \label{fig:R_1_of_z_zoom}
\end{figure}

\newpage

\section{Compering The Results}
One can that the principal radius parallel to the flow is much larger then the principal radius normal to the flow, which means $R_2\rightarrow\infty$. The derived equation for the radius is therefore:
\begin{equation}
    \begin{array}{rcl}
        \displaystyle\frac{a^2}{R_1^2}&=&\displaystyle\sqrt{1+\frac{2z}{Fr^2a}+\frac{2}{We}\left(1-\frac{a}{R_1}\right)} \\\\
        \displaystyle\frac{a^4}{R_1^4}&=&\displaystyle1+\frac{2z}{Fr^2a}+\frac{2}{We}-\frac{2}{We}\frac{a}{R_1} \\\\
        a^4&=&\displaystyle R^4+\frac{2zR_1^4}{Fr^2a}+\frac{2R_1^4}{We}-\frac{2aR_1^3}{We} \\\\
        0&=&\displaystyle\left(1+\frac{2z}{Fr^2a}+\frac{2}{We}\right)R_1^4-\frac{2aR_1^3}{We}-a^4
    \end{array}
\end{equation}
Using the \emph{roots} function in \emph{MatLab} we can claculate the radius $R_1$ as a function of z
\begin{figure}[H]
    \centering
    \includegraphics[width=1\textwidth]{images/graph3.png}
    \caption{$R_1$ as a function of z when $R_2\rightarrow\infty$}
    \label{fig:R_1_of_z_when_R_2_inf}
\end{figure}
We can see that the approximate analytical solution follows closely the numerical solution. \\

Another assumption one can make is to assume that the Webber number is large so $\frac{1}{We}\rightarrow0$
\begin{figure}[H]
    \centering
    \includegraphics[width=1\textwidth]{images/graph4.png}
    \caption{$R_1$ as a function of z when $We\rightarrow\infty$}
    \label{fig:R_1_of_z_when_We_inf}
\end{figure}
We can see that this assumption is not good and falls apart when z (U) is large enough.

\newpage

\section{Oscillatory Behavior}
We can see oscillations around the solution. The oscillations occur no matter the step size or the length of the solution. I think that the oscillations are not a result of the numerical method but they are a part of the actual solution. They describe the balance between the acceleration due to gravity and the deceleration due to surface tension. Moreover, the oscillations don't occur when the principal radius parallel to the flow is neglected, which strengthens the fact that the oscillations are the results of the balance between the curvature and gravity.

\newpage

\appendix
\section{The Code}
\begin{lstlisting}[ frame=single, numbers=left, style=MatLab-editor]
clc; clear; close all;

U0 = 1; % [m/sec]
a = 0.05; % [m]
rho = 998; % [kg/m^3]
sigma = 0.0728; % [N/m]
g = 9.81; % [m/sec^2]
z_max = 5; % [m]

We = rho * U0^2 * a / sigma;
Fr = U0^2 / g / a;

h = 1e-5; % [m]

zs_temp = 0:h:z_max;
for i = 1:length(zs_temp)
    zs(i,1) = zs_temp(i);
end
xs = zeros(length(zs), 1);
ys = zeros(length(zs), 1);

xs(1) = a;
ys(1) = 0;

for i = 1:length(zs)-1
    m1 = h * calc_g(zs(i), xs(i), ys(i));
    k1 = h * calc_f(zs(i), xs(i), ys(i), We, Fr, a);
    m2 = h * calc_g(zs(i) + 1/3*h, xs(i) + 1/3*m1, ys(i) + 1/3*k1);
    k2 = h * calc_f(zs(i) + 1/3*h, xs(i) + 1/3*m1, ys(i) + 1/3*k1, We, Fr, a);
    m3 = h * calc_g(zs(i) + 1/3*h, xs(i) + 1/6*(m1+m2), ys(i) + 1/6*(k1+k2));
    k3 = h * calc_f(zs(i) + 1/3*h, xs(i) + 1/6*(m1+m2), ys(i) + 1/6*(k1+k2), We, Fr, a);
    m4 = h * calc_g(zs(i) + 1/2*h, xs(i) + 1/8*(m1+3*m3), ys(i) + 1/8*(k1+3*k3));
    k4 = h * calc_f(zs(i) + 1/2*h, xs(i) + 1/8*(m1+3*m3), ys(i) + 1/8*(k1+3*k3), We, Fr, a);
    m5 = h * calc_g(zs(i) + h, xs(i) + 1/2*(m1-3*m3+4*m4), ys(i) + 1/2*(k1-3*k3+4*k4));
    k5 = h * calc_f(zs(i) + h, xs(i) + 1/2*(m1-3*m3+4*m4), ys(i) + 1/2*(k1-3*k3+4*k4), We, Fr, a);

    xs(i+1) = xs(i) + 1/6*(m1 + 4*m4 + m5);
    ys(i+1) = ys(i) + 1/6*(k1 + 4*k4 + k5);
end

R1s = xs;

%% Q4
fig1 = figure ("Name","Jet Shape",'Position',[100 300 900 500]);
colors = cool(4)*0.9;
hold all

plot(R1s, z_max-zs, -R1s, z_max-zs, "-", "LineWidth", 1.5, "Color", colors(1,:))

xlabel('R_1 [m]')
ylabel('z [m]')
grid on
grid minor
title("Jet Shape")
subtitle("Almog Dobrescu 214254252")
% legend({},'FontSize',11 ,'Location','northwest')
% exportgraphics(fig1, 'graph1.png','Resolution',300);

fig2 = figure ("Name","R1 as a Function of z",'Position',[250 300 900 500]);
colors = cool(4)*0.9;
hold all

plot(zs, R1s, "-", "LineWidth", 1.5, "Color", colors(1,:))

xlabel('z [m]')
ylabel('R_1 [m]')
grid on
grid minor
title("R1 as a Function of z")
subtitle("Almog Dobrescu 214254252")
% legend({},'FontSize',11 ,'Location','northwest')
% exportgraphics(fig2, 'graph2.png','Resolution',300);

%% Q5

R1s_when_R2_is_inf = zeros(length(zs), 1);
for i = 1:length(zs)
    radiuses = roots([1+2*zs(i)/Fr^2/a+2/We, -2*a/We, 0, 0, -a^4]);
    for index = 1:length(radiuses) 
        if (isreal(radiuses(index)) && 0<radiuses(index))
            R1s_when_R2_is_inf(i,1) = radiuses(index);
        end
    end
end

fig3 = figure ("Name","R1 numerical vs R1 when R2 is inf",'Position',[400 300 900 500]);
colors = cool(4)*0.9;
hold all

plot(zs, R1s, "-", "LineWidth", 3, "Color", colors(1,:))
plot(zs, R1s_when_R2_is_inf, "-", "LineWidth", 1.5, "Color", colors(4,:))

xlabel('z [m]')
ylabel('R_1 [m]')
grid on
grid minor
title("R1 numerical vs R1 when R2 is inf")
subtitle("Almog Dobrescu 214254252")
legend({'Numerical solution','Apprximate analytical solution'},'FontSize',11 ,'Location','northeast')
% exportgraphics(fig3, 'graph3.png','Resolution',300);

fig4 = figure ("Name","R1 numerical vs R1 when We is inf",'Position',[550 300 900 500]);
colors = cool(4)*0.9;
hold all

plot(zs, R1s, "-", "LineWidth", 1.5, "Color", colors(1,:))
plot(zs, a.*(1+2.*g.*zs./U0^2).^(-1/4), "-", "LineWidth", 1.5, "Color", colors(4,:))

xlabel('z [m]')
ylabel('R_1 [m]')
grid on
grid minor
title("R1 numerical vs R1 when We is inf")
subtitle("Almog Dobrescu 214254252")
legend({'Numerical solution','Apprximate analytical solution'},'FontSize',11 ,'Location','northeast')
% exportgraphics(fig4, 'graph4.png','Resolution',300);

%% functions
function g = calc_g(z, x, y)
    g = y;
end

function f = calc_f(z, x, y, We, Fr, a)
    f = We*a^3/2/x^4 - We/2/a - We*z/Fr^2/a^2 - 1/a + 1/x;
end
\end{lstlisting}

\end{document}